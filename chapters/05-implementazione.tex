\section{Quinto Capitolo: Implementazione}

\subsection{Architettura del Sistema}

\blindtext[2]

\begin{figure}[H]
    \centering
    \includegraphics[width=0.95\textwidth]{example-image-c}
    \caption{Schema completo dell'architettura implementata}
    \label{fig:architettura-completa}
\end{figure}

\subsection{Componenti Backend}

\blindtext[1]

\subsubsection{Gestione Dati}

\blindtext[2]

\subsubsection{API e Servizi}

\blindtext[1]

\subsection{Componenti Frontend}

\blindtext[1]

\subsubsection{Interfaccia Utente}

\blindtext[2]

\begin{figure}[H]
    \centering
    \includegraphics[width=0.8\textwidth]{example-image-b}
    \caption{Mockup dell'interfaccia principale}
    \label{fig:interfaccia-principale}
\end{figure}

\subsubsection{Gestione dello Stato}

\blindtext[1]

\subsection{Integrazione e Deploy}

\blindtext[2]

\subsubsection{Pipeline CI/CD}

\blindtext[1]

Fasi della pipeline:
\begin{enumerate}
    \item Build e compilazione
    \item Testing automatizzato
    \item Quality checks
    \item Deploy su staging
    \item Deploy su produzione
\end{enumerate}

\subsubsection{Configurazione Ambiente}

\blindtext[1]

\subsection{Sicurezza Implementata}

\blindtext[2]

Misure di sicurezza adottate:
\begin{itemize}
    \item Autenticazione multi-fattore
    \item Crittografia end-to-end
    \item Rate limiting
    \item Input validation
    \item Security headers
\end{itemize}

\subsection{Ottimizzazioni}

\blindtext[1]

\subsubsection{Performance}

\blindtext[1]

\subsubsection{Caching}

\blindtext[1]

\subsection{Sintesi del Capitolo}

\blindtext[1]
