\appendix

\section{Appendice A: Codice Sorgente}

\blindtext[1]

\subsection{Esempio di Codice}

Di seguito un esempio di codice sorgente:

\begin{lstlisting}[language=Python, caption=Esempio di funzione Python]
def esempio_funzione(parametro):
    """
    Questa e' una funzione di esempio.
    
    Args:
        parametro: Descrizione del parametro
        
    Returns:
        Risultato dell'elaborazione
    """
    risultato = parametro * 2
    return risultato

# Utilizzo della funzione
valore = esempio_funzione(42)
print(f"Risultato: {valore}")
\end{lstlisting}

\blindtext[1]

\section{Appendice B: Dati Sperimentali}

\blindtext[2]

\subsection{Tabella dei Risultati}

\begin{table}[H]
\centering
\caption{Esempio di tabella con risultati sperimentali}
\label{tab:risultati-esempio}
\begin{tabular}{@{}lcccc@{}}
\toprule
\textbf{Metrica} & \textbf{Test 1} & \textbf{Test 2} & \textbf{Test 3} & \textbf{Media} \\ 
\midrule
Performance (ms) & 120 & 115 & 118 & 117.7 \\
Throughput (req/s) & 850 & 870 & 860 & 860.0 \\
Error Rate (\%) & 0.1 & 0.2 & 0.1 & 0.13 \\
CPU Usage (\%) & 45 & 48 & 46 & 46.3 \\
\bottomrule
\end{tabular}
\end{table}

\blindtext[1]

\section{Appendice C: Configurazioni}

\blindtext[1]

\subsection{File di Configurazione}

\begin{lstlisting}[caption=Esempio di file di configurazione]
{
  "nome_progetto": "Template Demo",
  "versione": "1.0.0",
  "configurazione": {
    "ambiente": "produzione",
    "porta": 8080,
    "database": {
      "host": "localhost",
      "porta": 5432,
      "nome": "demo_db"
    }
  },
  "features": {
    "cache": true,
    "logging": true,
    "monitoring": true
  }
}
\end{lstlisting}

\blindtext[1]