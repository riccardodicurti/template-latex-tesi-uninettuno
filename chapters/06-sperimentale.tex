\section{Sesto Capitolo: Valutazione Sperimentale}

\subsection{Metodologia di Testing}

\blindtext[2]

\subsection{Metriche di Valutazione}

\blindtext[1]

Metriche principali:
\begin{itemize}
    \item \textbf{Performance}: Tempo di risposta, throughput
    \item \textbf{Scalabilità}: Utenti concorrenti supportati
    \item \textbf{Affidabilità}: Uptime, error rate
    \item \textbf{Usabilità}: Task completion rate, soddisfazione utente
\end{itemize}

\subsection{Test di Performance}

\blindtext[2]

\subsubsection{Risultati Test di Carico}

\blindtext[1]

\begin{figure}[H]
    \centering
    \includegraphics[width=0.85\textwidth]{example-image-a}
    \caption{Grafico dei risultati dei test di carico}
    \label{fig:test-carico}
\end{figure}

\subsubsection{Analisi dei Tempi di Risposta}

\blindtext[1]

\subsection{Test di Usabilità}

\blindtext[2]

\begin{figure}[H]
    \centering
    \includegraphics[width=0.75\textwidth]{example-image-b}
    \caption{Risultati dei test di usabilità}
    \label{fig:test-usabilita}
\end{figure}

\subsection{Confronto con Soluzioni Esistenti}

\blindtext[2]

\subsubsection{Analisi Comparativa}

\blindtext[1]

\subsection{Discussione dei Risultati}

\blindtext[2]

Principali evidenze:
\begin{itemize}
    \item Miglioramento significativo delle performance
    \item Riduzione dei costi operativi
    \item Maggiore scalabilità del sistema
    \item Feedback positivo dagli utenti
\end{itemize}

\subsection{Limitazioni dello Studio}

\blindtext[1]

Limitazioni identificate:
\begin{itemize}
    \item Dimensione del campione
    \item Durata del periodo di test
    \item Contesto specifico di applicazione
\end{itemize}

\subsection{Sintesi del Capitolo}

\blindtext[1]